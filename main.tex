%===============================================================
%========================== PACKAGES ===========================
%===============================================================

%%%%%%%%%%%%%%%%%%%%%%%%%%%%%%%%%%%%%%%%%%%%%%%%%%%%%
%%% https://www.youtube.com/watch?v=mZcV1wIPCBo   %%%
%%%%%%%%%%%%%%%%%%%%%%%%%%%%%%%%%%%%%%%%%%%%%%%%%%%%%
% use report for big docs
\documentclass[12pt,twoside]{report} 

\usepackage[utf8]{inputenc}

\usepackage{graphicx}
\graphicspath{{images/}}

% for margins - bindingoffset and twoside (see above) cooperate
\usepackage[a4paper,width=160mm,top=25mm,bottom=25mm,bindingoffset=3mm]{geometry}


% headings - footnotes stuff
\usepackage{fancyhdr}
\pagestyle{fancy}
\fancyhead{}                       % to clear everything prior to configuring
\fancyhead[LE,RO]{\leftmark}       % to add chapter in heading
\fancyhead[RE,LO]{\thepage}
\fancyfoot{}                       % to clear everything prior to configuring
\renewcommand{\headrulewidth}{0pt} % line width
\renewcommand{\footrulewidth}{0pt} % line width
\setlength{\headheight}{15pt}      % needed to avoid some warning about headheight

\usepackage{fvextra}               % needed to avoid warning from csquotes
\usepackage{csquotes}              % needed to avoid warning from babel

% for bibliography
\usepackage[sorting=none]{biblatex}
\addbibresource{references.bib}

% for clickable table of contents
\usepackage{color}
\usepackage{hyperref}
\hypersetup{
    colorlinks=true, % set true if you want colored links
    linktoc=all,     % set to all if you want both sections and subsections linked
    linkcolor=blue,  % choose some color if you want links to stand out
}

% for referencing see e.g "\label{sec:greek_abstract}"
\usepackage{nameref}   
\usepackage{tocbibind}

\usepackage[greek,english]{babel}

% for greek symbols in text mode
\usepackage{alphabeta} 

\setlength{\parindent}{0em}
\setlength{\parskip}{1em}

% Math
% https://www.overleaf.com/learn/latex/Mathematical_expressions
% https://en.wikibooks.org/wiki/LaTeX/Mathematics
\usepackage{amsmath,amsfonts,amssymb,mathtools}
\usepackage{gensymb}
\usepackage{textcomp}

\usepackage{lipsum}  
%===============================================================
%====================== DOCUMENT BEGINS ========================
%===============================================================
\begin{document}

% to avoid numbers and styling till introduction chapter
\pagestyle{empty}

% \begin{titlepage}
    \begin{center}
    
    \includegraphics[scale=0.35]{ntua_logo.jpg}
    
    \vspace{5mm}
    
    {\small NATIONAL TECHNICAL UNIVERSITY OF ATHENS}
    
    {\small SCHOOL OF MECHANICAL ENGINEERING}
    
    \vspace{5mm}
    
    {\footnotesize POSTGRADUATE PROGRAM ON AUTOMATION SYSTEMS}
    
    {\footnotesize CONTROL SYSTEMS LAB}
    
    \vspace{5mm}
    
    {\small THESIS PROJECT}
    
    \textbf{{\Huge Multicopter control using dynamic vision and neuromorphic computing}}

    \vspace{20mm}

    {\Large Ntouros Evangelos}
    
    \vspace{20mm}
    
    \textbf{Supervisor}: Prof. Kostas J. Kyriakopoulos 
    
    \vspace{20mm}
    
    Athens, June 2022
    
    \end{center}
% \end{titlepage}

% \begin{titlepage}
    \begin{center}
    
    \includegraphics[scale=0.35]{ntua_logo.jpg}
    
    \vspace{5mm}
    
    {\small ΕΘΝΙΚΟ ΜΕΤΣΟΒΙΟ ΠΟΛΥΤΕΧΝΕΙΟ}
    
    {\small ΣΧΟΛΗ ΜΗΧΑΝΟΛΟΓΩΝ ΜΗΧΑΝΙΚΩΝ}
    
    \vspace{5mm}
    
    {\footnotesize ΔΙΑΤΜΗΜΑΤΙΚΟ ΠΡΟΓΡΑΜΜΑ ΜΕΤΑΠΤΥΧΙΑΚΩΝ ΣΠΟΥΔΩΝ \\
                   ΣΥΣΤΗΜΑΤΑ ΑΥΤΟΜΑΤΙΣΜΟΥ}
    
    {\footnotesize ΕΡΓΑΣΤΗΡΙΟ ΑΥΤΟΜΑΤΟΥ ΕΛΕΓΧΟΥ}
    
    \vspace{5mm}
    
    {\small ΔΙΠΛΩΜΑΤΙΚΗ ΕΡΓΑΣΙΑ}
    
    \textbf{{\Huge Έλεγχος πολυκοπτέρου με χρήση νευρομορφικού υπολογιστικού συστήματος και με μετρήσεις δυναμικού αισθητήρα όρασης}}

    \vspace{10mm}

    {\Large Ντούρος Ευάγγελος}
    
    \vspace{20mm}
    
    \textbf{Επιβλέπων}: Κωστας Κυριακόπουλος, Καθηγητής Ε.Μ.Π.
    
    \vspace{20mm}
    
    Αθήνα, Ιούνιος 2022
    
    \end{center}
% \end{titlepage}


\vspace{20mm}

.................................

Ntouros Evangelos, Electrical and Computer Engineer

\vspace{20mm}

Copyright etc

\tableofcontents

%fancy style from now on
\pagestyle{fancy}

\chapter*{Περίληψη}
\label{sec:greek_abstract}
\addcontentsline{toc}{chapter}{\nameref{sec:greek_abstract}}


\chapter*{Abstract}
\label{sec:abstract}
\addcontentsline{toc}{chapter}{\nameref{sec:abstract}}


\chapter*{Ευχαριστίες}
\label{sec:greek_ack}
\addcontentsline{toc}{chapter}{\nameref{sec:greek_ack}}


\chapter*{Acknowledgment}
\label{sec:ack}
\addcontentsline{toc}{chapter}{\nameref{sec:ack}}


\chapter*{Glossary}
\label{sec:glossary}
\addcontentsline{toc}{chapter}{\nameref{sec:glossary}}


\chapter*{List of Acronyms}
\label{sec:acronyms}
\addcontentsline{toc}{chapter}{\nameref{sec:acronyms}}


\chapter*{List of Figures}
\label{sec:figures}
\addcontentsline{toc}{chapter}{\nameref{sec:figures}}


\chapter*{List of Tables}
\label{sec:tables}
\addcontentsline{toc}{chapter}{\nameref{sec:tables}}


\chapter{Introduction}
This is a citation \cite{dummykey}

This is also a citation \parencite[e.g.][page 213]{dummykey}

\lipsum[1-10]


\chapter{First Chapter}

\section{First section}

\lipsum[1-10]

\section{Second section}

\lipsum[1-10]


\chapter{Conclusion}

\section{First section}

\lipsum[1-10]

\section{Second section}

\lipsum[1-10]


\appendix
\chapter{FPDA Design}
\input{appendix_a.tex}

\chapter{PyNN scripts}
\input{appendix_b.tex}

\printbibliography

\end{document}
